\documentclass[11pt,a4paper, 
swedish, english %% Make sure to put the main language last!
]{article}
\pdfoutput=1

%% Andréas's custom package 
%% (Will work for most purposes, but is mainly focused on physics.)
\usepackage{../custom_as}

%% Figures can now be put in a folder: 
\graphicspath{ {figures/} %{some_folder_name/}
}

%% If you want to change the margins for just the captions
\usepackage[margin=10 pt]{caption}

%% To add todo-notes in the pdf
\usepackage[%disable  %%this will hide all notes
]{todonotes} 

%% Cange the margin in the documents
\usepackage[
%            top    = 3cm,              %% top margin
%            bottom = 3cm,              %% bottom margin
%            left   = 3cm, right  = 3cm %% left and right margins
]{geometry}


%% If you want to chage the formating of the section headers
%\renewcommand{\thesection}{...}

\swapcommands{\Lambda}{\varLambda}
\swapcommands{\Omega}{\varOmega}
\swapcommands{\Gamma}{\varGamma}

%%%%%%%%%%%%%%%%%%%%%%%%%%%%%%%%%%%%%%%%%%%%%%%%%%%%%%%%%%%%%%%%%%%%%%
\begin{document}%% v v v v v v v v v v v v v v v v v v v v v v v v v v
%%%%%%%%%%%%%%%%%%%%%%%%%%%%%%%%%%%%%%%%%%%%%%%%%%%%%%%%%%%%%%%%%%%%%%
\title{Some rejected results\\
{\Large Theoretical results with no immediate application to the main project}}
\author{Andréas Sundström}
\date{\today}

\maketitle
%%%%%%%%%%%%%%%%%%%%%%%%%%%%%%%%%%%%%%%%%%%%%%%%%%%%%%%%%%%%%%%%%%%%%%

\section*{Introduction}
In this document are a few theoretical results derived during the
project on hyperthermia, in August 2017. The results in here are
however not immediately applicable to the main goal of the project, but
keept here because they might still be interesting in and of
themselves. 



\section{The Fourier coefficients of the frequency domain 
circle ring kernel}
This problem stems from the want to find the Fourier
cosine\footnotemark{} coefficients of  
\begin{equation}
K_S(\Delta\theta) = \frac{R}{2\pi}\,\frac{J_1\qty(RQ)}{Q}
=\sum_{n=0}^\infty A_{n} \, \cos(n\phi)
\end{equation}
where $Q=\sqrt{k^2+{k'}^2-2kk'\cos(\phi)}$. 
\footnotetext{We only have cosines in the series, since $K_S(\phi)$ is
even.}
This was because we wanted to find the eigenvalues to 
\begin{equation}\label{eq:eigint-pol}
\lambda\psi(k, \theta) =
\int_{q\Gamma}^\Gamma\rd{k'}\,k'
\int_{0}^{2\pi}\rd\theta'\,K_S(k, k', \theta-\theta') \psi(k',\theta'),
\end{equation}
and if we expand both $K_S$ and $\psi$ in Fourier series, we can
eliminate the $\theta'$ integral.

It just so happens to excist an expansion 
\cite[formula 8.532.1]{Gradshteyn-Ryzhik} for preciecely this type of
expression:
\begin{equation}
\frac{J_\nu\qty(RQ)}{Q^\nu} 
= 2^\nu R^{-\nu} \varGamma(\nu) \sum_{m=0}^\infty ?? (m+\nu) 
\frac{J_{m+\nu}(Rk)\,J_{m+\nu}(Rk')}{(kk')^\nu} 
C_m^{(\nu)}\Big(\cos(\phi)\Big),
\end{equation}
with $Q=\sqrt{k^2+{k'}^2-2kk'\cos(\phi)}$ exactly as above,
and where $C_m^{(\nu)}(x)$ are the Gegenbauer polynomials. Next the
gegenbauer polynomials reduce to the Chebychev polynomials of the
second kind for $\nu=1$: \cite[chapter 18.4]{Arfken-Weber}  
\begin{equation}
C_m^{(1)}(z) = U_m(z).
\end{equation}

We can also relate the Chebychev polynomials of the second kind to the
same of the first kind, $T_m(z)$, by\todo{cite!}
\begin{equation}\label{eq:U2T}
U_N(z) = 
\begin{cases}
2\sum_{j=1}^n T_{2j-1}(z)\qcomma&N=2n-1,\\
1+2\sum_{j=1}^n T_{2j}(z)\qcomma&N=2n.
\end{cases}
\end{equation}
This is useful because the Chebychev polynomials of the first kind
satisfy the identity \cite[chapter 18.4]{Arfken-Weber}
\begin{equation}\label{eq:Tcos}
T_m(\cos\phi) = \cos(m\phi).
\end{equation}
% Furthermore, for $z=\cos\phi$, we have
% \begin{equation}
% U_m(\cos\phi) = \frac{\sin[(n+1)\phi]}{\sin\phi}.
% \end{equation}

Now back to the kernel in question
\begin{equation}\label{eq:KS-U}
K_S(\phi) = \frac{R}{2\pi}\,\frac{J_1\qty(RQ)}{Q}
=\frac{1}{\pi kk'} \sum_{m=1}^\infty m
J_m(Rk)J_m(Rk') U_{m-1}\Big(\cos(\phi)\Big).
\end{equation}
To find the pure $\cos(\phi)$ term, we use the relation
\eqref{eq:U2T} but only keep the $T_1$ terms since those are the ones
that give $\cos(\phi)$. By \eqref{eq:U2T} we see that the
only terms containing $T_1(\cos(\phi))$ in \eqref{eq:KS-U} are
the ones where $m-1$ is odd, that is $m=2l$ has to be even, and where
$(m-1)\ge1$. We therefore get
\begin{equation}
A_1 = \frac{1}{\pi kk'} \sum_{l=1}^\infty 2l
J_{2l}(Rk)J_{2l}(Rk') \times2
=\frac{4}{\pi kk'} \sum_{l=1}^\infty l
J_{2l}(Rk)J_{2l}(Rk') 
\end{equation}
This can be generalized we can find any of the Fourier coefficients, for
$\cos(N\phi)$, by identifying all terms containing $T_N$. With
\eqref{eq:U2T}, it is clear that the indeces in \eqref{eq:KS-U} has to
satisfy $(m-1)\ge N$ and also $(m-1)$ has to have the same parity as
$N$, that is if $N$ is even then $m$ has to be odd and vice verse. We
are now ready to write down the coefficients by the parity of $N$. 
If $N=2n-1$ is odd, then
\begin{equation}
A_{2n-1} = \frac{2}{\pi kk'} 
\sum_{l=n}^\infty 2l J_{2l}(Rk)J_{2l}(Rk'),
\end{equation}
and if $N=2n$ is even and $N\ge2$, then
\begin{equation}
A_{2n} = \frac{2}{\pi kk'} 
\sum_{l=n}^\infty (2l-1) J_{2l-1}(Rk)J_{2l-1}(Rk'),
\end{equation}
and the special case where $N=0$
\begin{equation}
A_{0} = \frac{2}{\pi kk'} 
\qty[\frac{1}{2}+\sum_{l=1}^\infty (2l-1) J_{2l-1}(Rk)J_{2l-1}(Rk')].
\end{equation}

\subsection{Recursive relation for the Fourier coefficients}
For simplicity define $a_N:=\pi kk'\,A_N/2$. Next we begin by looking
at
\begin{equation}
\begin{aligned}
a_0+a_1 =& 
\frac{1}{2}+\sum_{l=1}^\infty (2l-1) J_{2l-1}(Rk)J_{2l-1}(Rk')
+\sum_{l=1}^\infty 2l J_{2l}(Rk)J_{2l}(Rk')\\
=& \frac{1}{2} + \sum_{m=1}^\infty m J_{m}(Rk)J_{m}(Rk')
\end{aligned}
\end{equation}








%%%%%%%%%%%%%%%%%%%%%%%%%% The bibliography %%%%%%%%%%%%%%%%%%%%%%%%%%
%\newpage
%% This bibliography ueses BibTeX
\bibliographystyle{ieeetr}
\bibliography{references}%requires a file named 'references.bib'
%% Citations are as usual: \cite{example_article}





%%%%%%%%%%%%%%%%%%%%%%%%%%%%%%%%%%%%%%%%%%%%%%%%%%%%%%%%%%%%%%%%%%%%%%
\end{document}%% ^ ^ ^ ^ ^ ^ ^ ^ ^ ^ ^ ^ ^ ^ ^ ^ ^ ^ ^ ^ ^ ^ ^ ^ ^ ^ ^
%%%%%%%%%%%%%%%%%%%%%%%%%%%%%%%%%%%%%%%%%%%%%%%%%%%%%%%%%%%%%%%%%%%%%%
