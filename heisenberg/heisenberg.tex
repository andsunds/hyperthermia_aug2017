\documentclass[11pt,a4paper, 
english, swedish %% Make sure to put the main language last!
]{article}
\pdfoutput=1

%% Andréas's custom package 
%% (Will work for most purposes, but is mainly focused on physics.)
\usepackage{../custom_as}

%% Figures can now be put in a folder: 
\graphicspath{ {figurer/} %{some_folder_name/}
}

%% If you want to change the margins for just the captions
\usepackage[margin=10 pt]{caption}

%% To add todo-notes in the pdf
\usepackage[%disable  %%this will hide all notes
]{todonotes} 

%% Cange the margin in the documents
\usepackage[
%            top    = 3cm,              %% top margin
%            bottom = 3cm,              %% bottom margin
%            left   = 3cm, right  = 3cm %% left and right margins
]{geometry}


%% If you want to chage the formating of the section headers
%\renewcommand{\thesection}{...}



%%%%%%%%%%%%%%%%%%%%%%%%%%%%%%%%%%%%%%%%%%%%%%%%%%%%%%%%%%%%%%%%%%%%%%
\begin{document}%% v v v v v v v v v v v v v v v v v v v v v v v v v v
%%%%%%%%%%%%%%%%%%%%%%%%%%%%%%%%%%%%%%%%%%%%%%%%%%%%%%%%%%%%%%%%%%%%%%


%% If you want to use an external file for the title page
%\input{titlepages.tex}


%%%%%%%%%%%%%%%%%%%% vvv Internal title page vvv %%%%%%%%%%%%%%%%%%%%%
%\title{Heissenber's inequality -- applied to EM fields}
%\author{Niklas Renström}
%\date{\today}

%\maketitle


%%%%%%%%%%%%%%%%%%%% ^^^ Internal title page ^^^ %%%%%%%%%%%%%%%%%%%%%
%% If you want a list of all todos
%\todolist



\section{Heisenberg's inequality and its signifance for Electromagnetic waves}
The Fourier theory states that any function in space can be divided into a linear combination of plane waves 
\begin{equation*}
 f(\vb*x) = \int_{\vb*k}d\vb*k \widetilde{f(\vb*k)} \ee^{\ii( \vb*k\vdot\vb*x - \omega t)}
\end{equation*}
with different wave-vectors $\vb*k$. %wavelengths and directions of propagations. 
One can speak of different representations in the space ($\vb*x$) domain and the spacial frequency ($\vb*k$) domain respectively, with the Fourier transform as the link between them.

Heisenberg's uncertainty principle, a famous formula in quantum physics states that one cannot perfectly determine both the position and momentum of a free particle in space. This inequality however, is only based on the wave nature of particles and can be seen more as a mathematical inequality on Hilbert spaces than  physical one. As such it also has an equivalent with respect to Fourier transforms which says that the more localized a function is in the space domain the more spread out it must be in the spacial frequency domain.

From Maxwell's equations we know that if an electric field in a lossless media oscillates in time with a certain angular frequency $\omega$ then it  must also oscillate in space with a certain wavelength $\lambda=\epsilon \mu \omega 2\pi$. Since the wavelength is basically the inverse of the length of the wavvevector $\vb*k$, the Fourier transform would be limited to contain wavevectors with such length. Thus an electric field generated by a NB signal will be highly localized in the frequency domain which should lead to a big spread in the space domain. Therefore an UWB signal would lead to a much better focus which contradicts actual results. We believe that this contradiction is due to a lack of detail in the reasoning and to understand why one must ``put the words into numbers'' and clarify what the uncertainty principle actually says and how it can be applied to the case of Hyperthermia.

\subsection{Decomposition of eletromagnetic waves in frequency components}

Due to the linearity of Maxwell's equations, if the signal generating an EM-field varies harmonically in time so will the field,
\begin{equation*}
\vb*E(\vb*x,t)=\vb*E(\vb*x)\cos(\omega t).
\end{equation*}
this allows the use of phasors,
\begin{equation*}
\vb*E(\vb*x,t)=\Re\qty[\vb*E(\vb*x)\ee^{-\ii\omega t}].
\end{equation*}
In a lossless homogeneous media, with this notation, Maxwell's equations lead to
\begin{equation}
  \label{eq:wave_free}
  \laplacian  \vb*E + \frac{\omega^2}{v^2}\vb*E=0,
\end{equation} 
where the speed of propagation $v=\frac{1}{\sqrt{\mu \epsilon}}$ has been inroduced. This equation has a solution
\begin{equation*}
\vb*E(\vb*x)=\vb*A\ee^{i\vb*k \cdot \vb*x},
\end{equation*}
a plane wave were the wavevector $\vb*k$ is any vector such that $\abs{\vb*k}^2=$. Again, since Maxwell's equations are linear, any sum of such solutions is also a solution and the general solution can be written as \todo{Kolla igenom och korrigera ekvationerna}
\begin{equation}
  \label{eq:gen_sol}
  \vb*E(\vb*x)=\int_{\abs{\vb*k}=\frac{\omega}{v}}\vb*A(\vb*k)\ee^{i\vb*k\cdot\vb*x}d\vb*k,
\end{equation}
where the amplitude $\vb*A(\vb*k)$ now represents the amplitude of each wavecomponent $\ee^{\ii\vb*k \cdot \vb*x}$.

As a nice function in space one can of course also expand it with its Fourier Transform
\begin{equation}
  \label{eq:f_exp}
  \vb*E(\vb*x)=\int_{\vb*k}\id\vb*k \widetilde{\vb*E(\vb*k)} \ee^{\ii(\vb*k\cdot\vb*x - \omega t)}.
\end{equation}


Comparing the Fourier expansion (\ref{eq:f_exp}), an integral over the whole $\vb*k$-space, with the general solution (\ref{eq:gen_sol}), an integral only over the subspace where $\abs{\vb*k}=\frac{\omega}{v}$ one can identify an expression for the Fourier transform $\widetilde{\vb*E(\vb*k)}$ as

\begin{equation}
  \label{eq:f-transf}
  \widetilde{\vb*E(\vb*k)}=\vb*A(\vb*k)\delta(\abs{\vb*k}-\frac{\omega}{v})=\vb*A(\theta,\phi)\delta(\abs{\vb*k-\frac{\omega}{v}}).
\end{equation}

Here a $\delta(\abs{\vb*k}-\frac{\omega}{v})$-term has been introduced to display the fact that the Fourier transform $\widetilde{\vb*E(\vb*k)}$ vanishes where $\abs{\vb*k} \neq\frac{\omega}{v}$ which is the consequence of the wave-equation \ref{eq:wave-free}.
This is further emphasized in the last step where one sees the amplitude $\vb*A$ not as a function of the wavevector $\vb*k$ but as a function of the angles in the $\vb*k$-space $\theta$ and $\phi$.

In short, for each frequency $\omega$ the Fourier transform of the EM-field is concentrated to a sphere shell with radius $\abs{\vb*k}=\frac{\omega}{v}$ and vanishes elsewhere.

\subsection{Heisenberg}
The uncertainty principle states that, for any sufficiently nice vector field, \todo{Kolla upp konstanten}
\begin{equation}
\Delta_{\vb*a}\vb*f \Delta_{\vb*alpha} \widetilde{\vb*f} \geq \frac{1}{2}, \quad \forall \vb*a, \vb*\alpha
\end{equation}
where $\Delta_a\vb*f$ is a measure of how spread out $\vb*f$ is around a point $\vb*a$;
\begin{equation}
  \label{eq:spread_def}
\frac{\Delta_{\vb*a}\vb*f=\int_{\vb*x}\id\vb*x\abs{\vb*x-\vb*a}^2\abs{\vb*f(\vb*x)}}{\int_{\vb*x}\id\vb*x\abs{\vb*f(\vb*x)}}.
\end{equation}

This implies that the EM-field can only be as localized as the inverse of the spread for its Fourier transform
\begin{equation}
  \label{eq:ineq_final}
  \Delta_{\vb*a}\vb*E\geq \frac{1}{2\mathrm{Min}_\alpha\Delta_{\vb*\alpha}\widetilde{\vb*E}}, \quad \forall \vb*a.
\end{equation}
Since the fourier transform is concentrated on a sphere shell we achieve the minimum spread for $\alpha=0$ and using equations \ref{eq:spread_def} one gets
$\mathrm{Min}_\alpha\Delta_{\vb*\alpha}\widetilde{\vb*E}=\frac{\omega^2}{v^2}$ and \ref{eq:ineq_final} becomes
\begin{equation}
 \Delta_{\vb*a}\vb*E\geq \frac{v^2}{2\omega^2}
 \end{equation}







%%%%%%%%%%%%%%%%%%%%%%%%%% The bibliography %%%%%%%%%%%%%%%%%%%%%%%%%%
%\newpage
%% This bibliography ueses BibTeX
\bibliographystyle{ieeetr}
\bibliography{references}%requires a file named 'references.bib'
%% Citations are as usual: \cite{example_article}


%%%%%%%%%%%%%%%%%%%%%%%%%%%%%%%%%%%%%%%%%%%%%%%%%%%%%%%%%%%%%%%%%%%%%%
\end{document}%% ^ ^ ^ ^ ^ ^ ^ ^ ^ ^ ^ ^ ^ ^ ^ ^ ^ ^ ^ ^ ^ ^ ^ ^ ^ ^ ^
%%%%%%%%%%%%%%%%%%%%%%%%%%%%%%%%%%%%%%%%%%%%%%%%%%%%%%%%%%%%%%%%%%%%%%




%%%%  Some (useful) templates


%% På svenska ska citattecknet vara samma i både början och slut.
%% Använd två apostrofer: ''.


%% Including PDF-documents
\includepdf[pages={1-}]{filnamn.pdf} % NO blank spaces in the file name

%% Figures (pdf, png, jpg, ...)
\begin{figure}\centering
\centerline{ % centers figures larges than 1\textwidth
\includegraphics[width=.8\textwidth]{file_name.pdf}
}
\caption{}
\label{fig:}
\end{figure}

%% Figures from xfig's "Combined PDF/LaTeX"
\begin{figure}\centering
\resizebox{.8\textwidth}{!}{\input{file_name.pdf_t}}
\caption{}
\label{fig:}
\end{figure}


%% If you want to add something to the ToC
%% (Without having an actual header in the text.)
\stepcounter{section} %For example a 'section'
\addcontentsline{toc}{section}{\Alph{section}\hspace{8 pt}Labblogg} 

