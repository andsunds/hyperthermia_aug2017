\documentclass[11pt,a4paper, 
english, swedish %% Make sure to put the main language last!
]{article}
\pdfoutput=1

%% Andréas's custom package 
%% (Will work for most purposes, but is mainly focused on physics.)
\usepackage{../custom_as}

%% Figures can now be put in a folder: 
\graphicspath{ {figurer/} %{some_folder_name/}
}

%% If you want to change the margins for just the captions
\usepackage[margin=10 pt]{caption}

%% To add todo-notes in the pdf
\usepackage[%disable  %%this will hide all notes
]{todonotes} 

%% Cange the margin in the documents
\usepackage[
%            top    = 3cm,              %% top margin
%            bottom = 3cm,              %% bottom margin
%            left   = 3cm, right  = 3cm %% left and right margins
]{geometry}


%% If you want to chage the formating of the section headers
%\renewcommand{\thesection}{...}
\newcommand{\Lsq}[1]{\ensuremath{mathcal{L}^2_{#1}}}


%%%%%%%%%%%%%%%%%%%%%%%%%%%%%%%%%%%%%%%%%%%%%%%%%%%%%%%%%%%%%%%%%%%%%%
\begin{document}%% v v v v v v v v v v v v v v v v v v v v v v v v v v
%%%%%%%%%%%%%%%%%%%%%%%%%%%%%%%%%%%%%%%%%%%%%%%%%%%%%%%%%%%%%%%%%%%%%%


%% If you want to use an external file for the title page
%\input{titlepages.tex}


%%%%%%%%%%%%%%%%%%%% vvv Internal title page vvv %%%%%%%%%%%%%%%%%%%%%
%\title{Heissenber's inequality -- applied to EM fields}
%\author{Niklas Renström}
%\date{\today}

%\maketitle


%%%%%%%%%%%%%%%%%%%% ^^^ Internal title page ^^^ %%%%%%%%%%%%%%%%%%%%%
%% If you want a list of all todos
%\todolist



\section{Heisenberg's inequality and its signifance for Electromagnetic waves}
The Fourier theory states that any function in space can be divided into a linear combination of plane waves 
\begin{equation*}
 f(\vb*x) = \int_{\vb*k}d\vb*k \widetilde{f}(\vb*k) \ee^{\ii( \vb*k\vdot\vb*x - \omega t)}
\end{equation*}
with different wave-vectors $\vb*k$. %wavelengths and directions of propagations. 
One can speak of different representations in the space ($\vb*x$) domain and the spacial frequency ($\vb*k$) domain respectively, with the Fourier transform as the link between them.

Heisenberg's uncertainty principle, a famous formula in quantum physics states that one cannot perfectly determine both the position and momentum of a free particle in space. This inequality however, is only based on the wave nature of particles and can be seen more as a mathematical inequality on Hilbert spaces than  physical one. As such it also has an equivalent with respect to Fourier transforms which says that the more localized a function is in the space domain the more spread out it must be in the spacial frequency domain.

From Maxwell's equations we know that if an electric field in a lossless media oscillates in time with a certain angular frequency $\omega$ then it  must also oscillate in space with a certain wavelength $\lambda=\sqrt{\epsilon \mu} \omega 2\pi$. Since the wavelength is basically the inverse of the length of the wavvevector $\vb*k$, the Fourier transform would be limited to contain wavevectors with such length. Thus an electric field generated by a NB signal will be highly localized in the frequency domain which should lead to a big spread in the space domain. Therefore an UWB signal would lead to a much better focus which contradicts actual results. We believe that this contradiction is due to a lack of detail in the reasoning and to understand why one must ``put the words into numbers'' and clarify what the uncertainty principle actually says and how it can be applied to the case of Hyperthermia.

\subsection{Decomposition of eletromagnetic waves in frequency components}

Due to the linearity of Maxwell's equations, if the signal generating an EM-field varies harmonically in time so will the field,
\begin{equation*}
\vb*E(\vb*x,t)=\vb*E(\vb*x)\cos(\omega t).
\end{equation*}
which allows the use of phasors,
\begin{equation*}
\vb*E(\vb*x,t)=\Re\qty[\vb*E(\vb*x)\ee^{-\ii\omega t}].
\end{equation*}
In a lossless homogeneous media, with this notation, Maxwell's equations lead to
\begin{equation}
  \label{eq:wave_free}
  \laplacian  \vb*E + \frac{\omega^2}{v^2}\vb*E=0,
\end{equation} 
where the speed of propagation $v=\frac{1}{\sqrt{\mu \epsilon}}$ has been inroduced. This equation has a solution
\begin{equation*}
\vb*E(\vb*x)=\vb*A\ee^{i\vb*k \cdot \vb*x},
\end{equation*}
a plane wave were the wavevector $\vb*k$ is any vector such that $\abs{\vb*k}^2=$. Again, since Maxwell's equations are linear, any sum of such solutions is also a solution and the general solution can be written as
\begin{equation}
  \label{eq:gen_sol}
  \vb*E(\vb*x)=\int_{\abs{\vb*k}=\frac{\omega}{v}}\vb*A(\vb*k)\ee^{i\vb*k\cdot\vb*x}\id^3k,
\end{equation}
where the amplitude $\vb*A(\vb*k)$ now represents the amplitude of each wavecomponent $\ee^{\ii\vb*k \cdot \vb*x}$.

As a nice function in space one can of course also expand it with its Fourier Transform
\begin{equation}
  \label{eq:f_exp}
  \vb*E(\vb*x)=\frac{1}{\2\pi}\int_{\vb*k} \widetilde{\vb*E}(\vb*k) \ee^{\ii(\vb*k\cdot\vb*x - \omega t)}\id^3k.
\end{equation}


Comparing the Fourier expansion (\ref{eq:f_exp}), an integral over the whole $\vb*k$-space, with the general solution (\ref{eq:gen_sol}), an integral only over the subspace where $\abs{\vb*k}=\frac{\omega}{v}$ one can identify an expression for the Fourier transform $\widetilde{\vb*E}(\vb*k)$ as

\begin{equation}
  \label{eq:f-transf}
  \widetilde{\vb*E}(\vb*k)=2\pi\vb*A(\vb*k)\delta(\abs{\vb*k}-\frac{\omega}{v})=2\pi\vb*A(\theta,\phi)\delta(\abs{\vb*k}-\frac{\omega}{v}).
\end{equation}

Here a $\delta(\abs{\vb*k}-\frac{\omega}{v})$-term has been introduced to display the fact that the Fourier transform $\widetilde{\vb*E}(\vb*k)$ vanishes where $\abs{\vb*k} \neq\frac{\omega}{v}$ which is the consequence of the wave-equation \ref{eq:wave-free}.
This is further emphasized in the last step where one sees the amplitude $\vb*A$ not as a function of the wavevector $\vb*k$ but as a function of the angles in the $\vb*k$-space $\theta$ and $\phi$.

In short, for each frequency $\omega$ the Fourier transform of the EM-field is concentrated to a sphere shell with radius $\abs{\vb*k}=\frac{\omega}{v}$ and vanishes elsewhere.

\subsection{Heisenberg's inequality}
The common version of Heisenberg's uncertainty principle used in quantum mechanics states that the product of the variance of two observables must be greater than the expected value of their commutator. With the correct interpretations this can be reduced to the following form about vector fields and their Fourier transform, 
\begin{equation}
\Delta_{\vb*a}f \Delta_{\vb*\alpha} \widetilde{f} \geq \frac{9}{4}, \quad \forall \vb*a, \vb*\alpha, \forall f \in \Lsq[\infty].
\end{equation}

$\Delta_af$ is a measure of how spread out $\f$ is around a point $\vb*a$;
\begin{equation}
  \label{eq:spread_def}
\Delta_{\vb*a}f=\frac{\int_{\vb*x}\id^3x\abs{\vb*x-\vb*a}^2\abs{f(\vb*x)}}{\int_{\vb*x}\id^3x\abs{f(\vb*x)}}.
\end{equation}
It achieves its minimum when $\vb*a$ is the center of mass of $f$ and is then directly analogous to the variance measure of probability theory.

\todo{EM-field not as a vector}
This implies that the EM-field can only be as localized as the inverse of the spread for its Fourier transform
\begin{equation}
  \label{eq:ineq_final}
  \Delta_{\vb*a}\vb*E\geq \frac{9}{4} \frac{1}{\mathrm{Min}_\alpha\Delta_{\vb*\alpha}\widetilde{\vb*E}}, \quad \forall \vb*a.
\end{equation}
Since the fourier transform of the EM-field is concentrated on a sphere shell we achieve the minimum spread for $\alpha=0$ and using equations \ref{eq:spread_def} one gets
$\mathrm{Min}_\alpha\Delta_{\vb*\alpha}\widetilde{\vb*E}=\frac{\omega^2}{v^2}$ and \ref{eq:ineq_final} becomes
\begin{equation}
 \Delta_{\vb*a}\vb*E\geq \frac{9v^2}{4\omega^2}=\frac{9\lambda^2}{16\pi^2}=\frac{9}{16\pi^2\epsilon\mu\nu^2}.
 \end{equation}

If the variation of $E$ is high it implies that $E$ cannot be highly localized to one small neighbourhood in space, but it does not preclude $E$ from being concentrated in small neighbourhoods of two or more widely spread out points. Therefore, for hyperthermia treatment, the variation measure at best gives a hint on what we want. We could achieve a very high peak in the tumour without violating the inequality as long as a significant part of the EM-field is located outside of the head.
This however is a consequence of the particular form we have stated the uncertainty relation, it does hold that a function cannot be highly localized both in the space and frequency domains but this can be quantified in many different ways, one of which we shall see in the next section.

\subsection{the Slepian-Landau-Pollak uncertainty relation}
In a series of papers from the 60's D.Slepian, H.J.Landau and H.O.Pollak investigated how much energy a frequency-limited function can keep if it is restricted in space. By doing this one can reach an inequality on the form
\begin{equation}
\abs{S}\abs{\Omega} \geq c(\alpha,\beta),
\end{equation}
Where $S$ and $\Omega$ are the areas the function is restricted to in space and frequency respectively and $\alpha$ and $\beta$ is the fraction of energy that the function contains in these areas. 

In what follows we will show the broad strokes for developing the theory from which the inequality can be extracted. In doing so we will prove the existance of a useful set of functions ${\psi_i}$ who span the room of frequency-limited functions and one which is the function retaining the most energy if it is limited to a certain space $S$.

A square-integrable function $f$ of $D$ variables is said to be $\Omega$-limited if it can be represented as a Fourier integral over the bounded region $\Omega$,

\begin{equation}
  f(\id*x)=\qty(\frac{1}{2\pi})^D\int_\Omega \widetilde{f}(\id*k)\ee^{\ii(\id*k \cdot \id*x)} \id^Dk,
\end{equation}
that is, it only contains frequencies in the area $\Omega$.
We define the energy of a function to be
\begin{equation*}
A=\int_{R^D}\abs{f(\id*x)}^2\id^3x.
\end{equation*}
By Parseval's theorem we get that
\begin{equation*}

  \end{equation*}



%%%%%%%%%%%%%%%%%%%%%%%%%% The bibliography %%%%%%%%%%%%%%%%%%%%%%%%%%
%\newpage
%% This bibliography ueses BibTeX
\bibliographystyle{ieeetr}
\bibliography{references}%requires a file named 'references.bib'
%% Citations are as usual: \cite{example_article}


%%%%%%%%%%%%%%%%%%%%%%%%%%%%%%%%%%%%%%%%%%%%%%%%%%%%%%%%%%%%%%%%%%%%%%
\end{document}%% ^ ^ ^ ^ ^ ^ ^ ^ ^ ^ ^ ^ ^ ^ ^ ^ ^ ^ ^ ^ ^ ^ ^ ^ ^ ^ ^
%%%%%%%%%%%%%%%%%%%%%%%%%%%%%%%%%%%%%%%%%%%%%%%%%%%%%%%%%%%%%%%%%%%%%%




%%%%  Some (useful) templates


%% På svenska ska citattecknet vara samma i både början och slut.
%% Använd två apostrofer: ''.


%% Including PDF-documents
\includepdf[pages={1-}]{filnamn.pdf} % NO blank spaces in the file name

%% Figures (pdf, png, jpg, ...)
\begin{figure}\centering
\centerline{ % centers figures larges than 1\textwidth
\includegraphics[width=.8\textwidth]{file_name.pdf}
}
\caption{}
\label{fig:}
\end{figure}

%% Figures from xfig's "Combined PDF/LaTeX"
\begin{figure}\centering
\resizebox{.8\textwidth}{!}{\input{file_name.pdf_t}}
\caption{}
\label{fig:}
\end{figure}


%% If you want to add something to the ToC
%% (Without having an actual header in the text.)
\stepcounter{section} %For example a 'section'
\addcontentsline{toc}{section}{\Alph{section}\hspace{8 pt}Labblogg} 

