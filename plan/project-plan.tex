\documentclass[11pt,a4paper, 
english, swedish %% Make sure to put the main language last!
]{article}
\pdfoutput=1

%% Andréas's custom package 
%% (Will work for most purposes, but is mainly focused on physics.)
\usepackage{../custom_as}

%% Figures can now be put in a folder: 
\graphicspath{ {figurer/} %{some_folder_name/}
}

%% If you want to change the margins for just the captions
\usepackage[margin=10 pt]{caption}

%% To add todo-notes in the pdf
\usepackage[%disable  %%this will hide all notes
]{todonotes} 

%% Change the margin in the documents
\usepackage[
%            top    = 3cm,              %% top margin
%            bottom = 3cm,              %% bottom margin
%            left   = 3cm, right  = 3cm %% left and right margins
]{geometry}


%% If you want to chage the formating of the section headers
%\renewcommand{\thesection}{...}



%%%%%%%%%%%%%%%%%%%%%%%%%%%%%%%%%%%%%%%%%%%%%%%%%%%%%%%%%%%%%%%%%%%%%%
\begin{document}%% v v v v v v v v v v v v v v v v v v v v v v v v v v
%%%%%%%%%%%%%%%%%%%%%%%%%%%%%%%%%%%%%%%%%%%%%%%%%%%%%%%%%%%%%%%%%%%%%%



%%%%%%%%%%%%%%%%%%%% vvv Internal title page vvv %%%%%%%%%%%%%%%%%%%%%
%\begin{titlepage}
\title{Project plan -- \texttt{project lambda}}
\author{Nilkas Renström \and Andréas Sundström}
\date{\today}

\maketitle

%%%%%%%%%%%%%%%%%%%% ^^^ Internal title page ^^^ %%%%%%%%%%%%%%%%%%%%%
%% If you want a list of all todos
%\todolist



\section{Introduction}
%\todo[inline]{General intro to hyopthermia and freq dep.}

This project will study the efficacy of ultra-wideband (UWB) treatmants in comparison to narrowband (NB). Previous studies have moslty been conducted using numerical or experimantal methods. In this project however, we intend to investigate this subject from a more theoretical and mathematical stand point. 

Prior studies have found indications that an UWB approach does not significantly improve performance over NB. This project aims to find some theoretical grounds for this hypothesis. Of special interest is the apparent problem of focusing the NB radiation, arising from the Fourier version of the Heissenberg uncertainty principle. The goal of this project will be to explain why the Heissenberg inequality breaks down in a scenario like hyperthermia treatment. 


\subsection{Theoretical background for Frequency decomposition}
Any function in space can be divided into a linear combination of plane waves with different wave-vectors.%wavelengths and directions of propagations. 
One can speak of different representations in the space domain and the frequency domain respectively, with the Fourier transform as the link between them.

Heisenberg's inequality states that if a function is highly localised in one domain it must be very spread out in the other. In the case of hyperthermia we wish to concentrate the radiation in space, and thus a wide band of frequencies should therefore be prefered to a narrow one. This however seems to be contradicted in previous studies. 
To understand why this can be one must ''put the words into numbers'' and clarify what Heisenberg's inequality actually says. 

First of all, frequencies and wavelengths although often mixed, are not the same thing. Frequencies speak of oscillations in time while wavelengths of oscillations in space. 
In non-dispersive (frequency-indepenent) materials however, the temporal frequency is related to the wavelength via the simple dispersion relation $\lambda=vf$, where $\lambda$ is the wavelength, $v$ is the speed of propagation and $f$ is the frequency. 

%The frequency of the signal is not the same thing as the wavelength of the EM-waves produced. Frequencies speak of oscillations in time while wavelengths of oscillations in space. 

%To begin with we will make the simplification that the antennas transmits plane EM-waves and just look at the propagation in a homogenous material. In this case the link is the speed of light and will appear as a simple conversation factor. However there might still be difficulties due to traveling wave.





\section{Method and limitations}
As there have been litte to no previous theoretical studies on the subject of the nature of focusing microwaves inside tissue, this study will have to start with a fairly simple theoretical model -- concisting of simple homogeneuos regions with only one or two transitions.  
%From this we hope that future work will have firmer theoretical grounds when studying more complex systems;

Ideally the results from this study will be aplicable to a more realistic model of the head and neck region.
However, as this project is relatively short we will probably have to limit our analysis to the simple models presented below.  

\subsection{\textit{Part 1}}
\todo[inline]{Part 1 is simple models, explain how}
\subsubsection{Understanding the Heisenberg inequality}

To begin with we will use a very simple model of a completely homogeneous material in wich plane EM-waves propogate. Then the link between temporal frequency and wavelengths will be the speed of propogation, a simple conversion facctor.
We will then write and prove a version of the Heisenberg inequality which can be applied to this specific case and plug in the actual numbers, how big bandwidth does one need to focus for one centimeter? 
Pherhaps all that is needed is a bandwidth of a few kHz and the NB approach is perfectly justifiable. Or our results will show that a good focus would require a large band of frequencies, pherhaps even larger than the microwave spectra. 
From early estimates the latter is what we expect which however would contradict results from earlier numerical studies. There are several possible explanations which in this case we will investigate.

1. The field could have a periodic pattern and therefore not be highly localized. However if the wavelength of the pattern is greater than that of the head we could still achieve local hotspots only in the tumour.

2. The transition from one material to another plays a significal role for the focus of the EM-field and the homogeneous model is too simple. To investigate this the model will be extended to include material transitions, at first step by introducing a sphere of a different material in our previous homogeneous material. This is described more in detail in the following section.
\todo[inline]{Why doesn't Heisenberg work, klart?}


\todo[inline]{Study  vs } 
Another interesting point is that from what we understand so far the variable in Heisenberg's inequality is the bandwidth of the wavelengths and not their actual value. Thus EM-waves with wavelengths of 9-10 meters would be able to produce focus as exact as those with wavelengths of 0-1 meters.





\subsubsection{A simple theoretical model} %focusing effect of head geometry}
Here we will build a simple model to study the microwave-head interaction theoretically. Our hopes are that this model can give some theoretical insight in how some specific (NB) frequencies manages to create a localized focus inside the head. 

The model concists of a plane wave incident on a homogeneous sphere, with $\epsilon>1$ and $\sigma\neq0$. With this model we hope to find the behaviour of the wave inside the sphere. Of special interest here is the relation beween the wavelengths used and the radius of the sphere. Here these two will be of comparable length scales. \todo[]{De behöver inte vara jämförbara, behandlingen görs med microvågor, 400-800MHz som har en viss våglängd och utförs på huvud/näsa med en viss längdskala}

This is motivated by an idea that a head or a sphere, which are both convex, might have some lensing effect. The plane wave together with a sphere is the most simple non-trivial model conceivable, and might be analytically solvable. To the very least we should be able to find some integral formula which then can be calculated numerically.

Once \emph{one} plane wave has been solved, it should not be very hard to find the behavior from a superposition of plane waves incident from different directions -- but with the same temporal frequency. Hopefully this can give some focusing effect. 

If all of this works, then the same method might be applicable to more complex geometries. However, that will probably involve some integral formula over the geometry.



\subsection{\textit{Part 2}}
\todo[inline]{Explain part 2 - Amplitudes not just E-field, complex material transitions}





\section{Time line}
\todo[inline]{Tomorrow: finalize this document}


16th: have the theoretical work for "part 1" finished; decide whether or not to continue with \todo{Clarify what the two "parts" are.}"part 2".

17--21st: Hopefully we can have part 1 typed up during these days.


%\section{Further development}









%%%%%%%%%%%%%%%%%%%%%%%%%% The bibliography %%%%%%%%%%%%%%%%%%%%%%%%%%
%\newpage
%% This bibliography ueses BibTeX
%\bibliographystyle{ieeetr}
%\bibliography{references}%requires a file named 'references.bib'
%% Citations are as usual: \cite{example_article}

%%%%%%%%%%%%%%%%%%%%%%%%%%%%%%%%%%%%%%%%%%%%%%%%%%%%%%%%%%%%%%%%%%%%%%
\end{document}%% ^ ^ ^ ^ ^ ^ ^ ^ ^ ^ ^ ^ ^ ^ ^ ^ ^ ^ ^ ^ ^ ^ ^ ^ ^ ^ ^
%%%%%%%%%%%%%%%%%%%%%%%%%%%%%%%%%%%%%%%%%%%%%%%%%%%%%%%%%%%%%%%%%%%%%%



