\documentclass[11pt,a4paper, 
swedish, english %% Make sure to put the main language last!
]{article}
\pdfoutput=1

%% Andréas's custom package 
%% (Will work for most purposes, but is mainly focused on physics.)
\usepackage{../custom_as}

%% Figures can now be put in a folder: 
\graphicspath{ {figures/} %{some_folder_name/}
}

%% If you want to change the margins for just the captions
\usepackage[margin=10 pt]{caption}

%% To add todo-notes in the pdf
\usepackage[%disable  %%this will hide all notes
]{todonotes} 

%% Cange the margin in the documents
\usepackage[
%            top    = 3cm,              %% top margin
%            bottom = 3cm,              %% bottom margin
%            left   = 3cm, right  = 3cm %% left and right margins
]{geometry}


%% If you want to chage the formating of the section headers
%\renewcommand{\thesection}{...}

\swapcommands{\Lambda}{\varLambda}
\swapcommands{\Omega}{\varOmega}
\swapcommands{\Gamma}{\varGamma}

%%%%%%%%%%%%%%%%%%%%%%%%%%%%%%%%%%%%%%%%%%%%%%%%%%%%%%%%%%%%%%%%%%%%%%
\begin{document}%% v v v v v v v v v v v v v v v v v v v v v v v v v v
%%%%%%%%%%%%%%%%%%%%%%%%%%%%%%%%%%%%%%%%%%%%%%%%%%%%%%%%%%%%%%%%%%%%%%
\cite{PSWF-I_1961, PSWF-II_1961, PSWF-III_1962, PSWF-IV_1964, PSWF-V_1978}

\subsection{Finite thickness circle ring in the frequency domain (2D)}
So far we have looked at areas in the space and frequency domains
which have the same shape, filled discs or spheres, but are scaled
versions of each other ($S=c\Omega$). This does however forces us to
include frequencies all the way down to 0, which is not an ideal
description of a real hyperthermia system -- perhaps operating in the
range from few hundered MHz up to a couple of GHz.

We first need to define the areas $S$ and $\Omega$ in the space and
frequency domain respectively. For the space domain we still want a
filed region without a gap in the center, but for the frequency domain
we want to exclude the center part of the region. Let $S$ be a filled
circle disc of radius $R$, and $\Omega$ be the finite thickness circle
ring with outer radius $\Gamma$ and inner radius $q\Gamma$, $0\le q<1$.
The methodology used here is a direct translation of the ones used in
\cite{PSWF-I_1961} and \cite{PSWF-IV_1964} to $S$ and $\Omega$.


% As said before, $\Omega$ is a finite thickness circle ring with
% outer radius $\Gamma$ and inner radius $q\Gamma$. 
From the definition of the kernel $K_s$ in eqn. (6) in
\cite{PSWF-IV_1964}, we get 
\begin{equation}
\begin{aligned}
K_S(\vb*k-\vb*k') =& (2\pi)^{-2}
\oldint_S \rd^2r \ee^{\ii\vb*r\vdot\Delta\vb*k}\\
=& (2\pi)^{-2}
\int_0^{2\pi}\rd\theta \int_0^{R} \rd{r}\,r
 \ee^{\ii\, r\Delta{k}\,\cos\theta},
\end{aligned}
\end{equation}
where $\Delta{k} = \abs{\Delta\vb*k} = \abs{\vb*k-\vb*k'}$. To get to
the second step here, choose the integration axes so that $\Delta{k}$
lies on the $\theta=0$ axis and thus
$\vb*r\vdot\Delta\vb*k=r\Delta{k}\,\cos\theta$;
this can be done since, for the sake of the integration
$\Delta{\vb*k}$ can be viewed as a constant. Next we use the usual
Bessel function expansion \cite[formula 8.551.4b]{Gradshteyn-Ryzhik}
\begin{equation}
\ee^{\ii z\cos\theta} = \sum_{m=-\infty}^\infty
\ii^mJ_m(z)\ee^{\ii m\theta}
\end{equation}
to write 
\begin{equation}
\begin{aligned}
K_S(\vb*k-\vb*k') =& (2\pi)^{-2}
\sum_{m=-\infty}^\infty \ii^m
\int_0^{2\pi}\rd\theta\, \ee^{\ii m\theta}
\int_0^{R} \rd{r}\,r J_m(r\Delta{k}).
\end{aligned}
\end{equation}
The $\theta$ integral is zero for all $m\neq0$ and $2\pi$ for $m=0$;
to solve the $r$ integral we use 
\cite[formula~8.472.3]{Gradshteyn-Ryzhik} 
\begin{equation}
\frac{1}{z}\dv{z}\qty[zJ_1(z)] = J_0(z).
\end{equation}
Together this leads to
\begin{equation}
\begin{aligned}
K_S(\vb*k-\vb*k') =& (2\pi)^{-2}
\,(2\pi)\,\frac{1}{(\Delta{k})^2}
\int_0^{R\Delta{k}} \rd{z}\,z\, J_0(z)\\
=& (2\pi)^{-1} \frac{1}{(\Delta{k})^2}
\Big[zJ_1(z)\Big]_{z=0}^{R\Delta k}
&=(2\pi)^{-1} \frac{R}{\Delta{k}} J_1(R\Delta{k}).
\end{aligned}
\end{equation}

The next step is to manipulate the actual integral eigenvalue equation
\todo{ref to some eqn. before}
\begin{equation}
\lambda \psi(\vb*k) = \oldint_\Omega \rd^2k'\,
%(2\pi)^{-1} \frac{R}{\Delta{k}} J_1(R\Delta{k}) 
K_S(\vb*k-\vb*k')\psi(\vb*k')
\end{equation}
which by polar coordiane substitution becomes
\begin{equation}
\lambda\psi(k, \theta) =
\frac{R}{2\pi}\,\int_0^{2\pi}\rd\theta'\,
\int_{q\Gamma}^\Gamma\rd{k'}\,k'\frac{J_1(R\Delta{k})}{\Delta{k}} \psi(k',\theta')
\end{equation}
We note that, by the law of cosines, 
$\Delta{k}=\abs{\vb*k-\vb*k'}=\sqrt{k^2+{k'}^2-2kk'\cos(\Delta\theta)}$, 
where $k=\abs{\vb*k}$, $k'=\abs{\vb*k'}$ and
$\Delta\theta=\theta-\theta'$ is the angle between $\vb*k$ and
$\vb*k'$. This means that 
\begin{equation}\label{eq:KS-long}
K_S(k, k', \Delta\theta) = \frac{R}{2\pi}\,
\frac{J_1\qty(R\sqrt{k^2+{k'}^2-2kk'\cos(\Delta\theta)})}
{\sqrt{k^2+{k'}^2-2kk'\cos(\Delta\theta)}}
\end{equation}
is $2\pi$-periodic in $\Delta\theta$, and thus has a Fourier series
expansion
\todo{change to sin and cos representation.}
\begin{equation} \label{eq:KS-FS}
K_S(k, k', \Delta\theta) 
= \sum_{n=-\infty}^\infty a_n(k, k') \ee^{\ii n\Delta\theta}
= \sum_{n=-\infty}^\infty a_n(k, k') \ee^{\ii n\theta}\ee^{-\ii n\theta'},
\end{equation}
where
\begin{equation} \label{eq:KS-FS_coef}
a_n(k, k') = \frac{1}{2\pi} \int_0^{2\pi}\rd\phi\,
K_S(k, k', \phi) \ee^{-\ii n\phi}.
\end{equation}
From the rightmost side of \eqref{eq:KS-FS} we see that $K_S$ is
$2\pi$-periodic in \emph{both} $\theta$ and $\theta'$, which means
that $\psi(k, \theta)$ must be $2\pi$-periodic in $\theta$ as well and
thus have a Fourier series expansion
\begin{equation} \label{eq:psi-FS}
\psi(k, \theta) 
= \sum_{n=-\infty}^\infty b_n(k) \ee^{\ii n\theta}.
\end{equation}



\section{Numerical implementation of the 
Landau-Pollack-Slepian Theory}
\newcommand{\varD}{\ensuremath{\mathcal{D}}}

In the previous section, we saw that the teoretical limit, to how much
of the energy in a function bandlimited to the frequency region
$\Omega$ can be conentrated in the real space region \varD, is set by
an integral equation of the form 
\begin{equation} \label{eq:int-eig}
\lambda \Psi(\kappa) 
= \oldint_{\Omega} \rd{k}\, K_\varD(\kappa, k) \Psi(k)
\qcomma \kappa\in\Omega.
\end{equation}
This section will provide a method to solve this type of integral
eigenvalue equation numerically.

\subsection{Discretization of the integral equation}
The integral is discretized by
\begin{equation} \label{eq:inttosum}
\begin{aligned}
\oldint_\Omega\rd{k} &\to \sum_{n=1}^N\Delta{k}
=\frac{\abs{\Omega}}{N} \sum_{n=1}^N,\\
\Psi(k) \to \Psi_n = \Psi(k_n) &\qcomma
K_\varD(\kappa, k) \to K_{m, n} = K_\varD(k_m, k_n).
\end{aligned}
\end{equation}
where $\abs{\Omega}$ is the size of the region $\Omega$.
%Next, $\Psi$ and $K_\varD$ are discretized via
Note that $\kappa \to k_m$ (and not $\kappa_m$) since both 
$k, \kappa\in\Omega$, and thus they are both discretized in the same
way. 
With these discretizations, \eqref{eq:int-eig} discretizes to
\begin{equation}
\lambda\Psi_m = \frac{1}{N} \sum_{n=1}^N K_{m, n} \Psi_n
\end{equation}
which is just a matrix eigenvalue equation
\begin{equation} \label{eq:mtx-eig}
\Lambda \Psi = \mathsf{K}\Psi,
\end{equation}
where $\Lambda = N\lambda$ and $\mathsf{K}$ is the $N\times N$ matrix
whose element $(m, n)$ is $K_{m, n}$. From here, there are many high
performance linear algebra libraries to find the eigenvalues and
eigenvector to \eqref{eq:mtx-eig} numerically. 

\subsubsection{Physical interpretation}
An interesting physical interpretation of the discretization is that
the operation \eqref{eq:inttosum} can be viewed as introducing
\begin{equation}
%K_\varD(\kappa, k) \to K_\varD(\kappa, k) \times
\Delta{k}\qty[\delta(k-k_0) + \delta(k-k_1) +\ldots+\delta(k-k_N)]
\end{equation}
into the integral, and effectively limiting the frequencydomain
$\Omega$ to the dicrete frequencies $k_n$. 

This idea could possibly be more cloesly related to reality, where the
differnt antennas only transmitts in certain dicrete frequencies. And
thus the discretized eigenvalue problem gives the theoretical maximum
energy in a bounded region in space from a signal restriced to the
chosen frequencies. 
It is however worth pointing out that this result has only been proven
for the 1D case\cite{PSWF-V_1978}, but there is no reason to believ
that the discetized version cannot be extended to higher dimension
like in the continuous case. 

\begin{comment}
\subsubsection{Improving this method using Simpson's rule}
\todo[inline]{\url{https://math.stackexchange.com/questions/898087/discretization-of-an-integral}\\
Will this even work?}
The main bottle neck in the numerical solution of \eqref{eq:int-eig}
is the eigenvalue and eigenvector calculation. It would therefore be
desirable to keep $N$ small. On the other hand, to accurately
approximate the integral using \eqref{eq:inttosum}, $N$ has to be
large. 
%This problem can be handled by changing the discretization method of
%the integral. 
The simple Riemann sum approximation, \eqref{eq:inttosum}, is one of
the most rudimentary integral discretizations. A good way to improve
numerical accuracy without sacrificing computation time in the
eigenvalue calculations, would be to use a better discretization
method. 

One such method is Simpson's rule (some times also refered to as
Simpson's $\nicefrac13$ rule)
\begin{equation}
\int_{k_{n-1}}^{k_{n+1}}f(k)\id{k} \approx 
\frac{\Delta{k}}{3} \Big[f(k_{n-1}) + 4f(k_n) + f(k_{n+1})\Big].
\end{equation}
\end{comment}








%%%%%%%%%%%%%%%%%%%%%%%%%% The bibliography %%%%%%%%%%%%%%%%%%%%%%%%%%
%\newpage
%% This bibliography ueses BibTeX
\bibliographystyle{ieeetr}
\bibliography{references}%requires a file named 'references.bib'
%% Citations are as usual: \cite{example_article}

%%%%%%%%%%%%%%%%%%%%%%%%%%%%%%%%%%%%%%%%%%%%%%%%%%%%%%%%%%%%%%%%%%%%%%
\end{document}%% ^ ^ ^ ^ ^ ^ ^ ^ ^ ^ ^ ^ ^ ^ ^ ^ ^ ^ ^ ^ ^ ^ ^ ^ ^ ^ ^
%%%%%%%%%%%%%%%%%%%%%%%%%%%%%%%%%%%%%%%%%%%%%%%%%%%%%%%%%%%%%%%%%%%%%%




%%%%  Some (useful) templates


%% På svenska ska citattecknet vara samma i både början och slut.
%% Använd två apostrofer: ''.


%% Including PDF-documents
\includepdf[pages={1-}]{filnamn.pdf} % NO blank spaces in the file name

%% Figures (pdf, png, jpg, ...)
\begin{figure}\centering
\centerline{ % centers figures larges than 1\textwidth
\includegraphics[width=.8\textwidth]{file_name.pdf}
}
\caption{}
\label{fig:}
\end{figure}

%% Figures from xfig's "Combined PDF/LaTeX"
\begin{figure}\centering
\resizebox{.8\textwidth}{!}{\input{file_name.pdf_t}}
\caption{}
\label{fig:}
\end{figure}


%% If you want to add something to the ToC
%% (Without having an actual header in the text.)
\stepcounter{section} %For example a 'section'
\addcontentsline{toc}{section}{\Alph{section}\hspace{8 pt}Labblogg} 

