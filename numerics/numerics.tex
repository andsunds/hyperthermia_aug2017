\documentclass[11pt,a4paper, 
swedish, english %% Make sure to put the main language last!
]{article}
\pdfoutput=1

%% Andréas's custom package 
%% (Will work for most purposes, but is mainly focused on physics.)
\usepackage{../custom_as}

%% Figures can now be put in a folder: 
\graphicspath{ {figures/} %{some_folder_name/}
}

%% If you want to change the margins for just the captions
\usepackage[margin=10 pt]{caption}

%% To add todo-notes in the pdf
\usepackage[%disable  %%this will hide all notes
]{todonotes} 

%% Cange the margin in the documents
\usepackage[
%            top    = 3cm,              %% top margin
%            bottom = 3cm,              %% bottom margin
%            left   = 3cm, right  = 3cm %% left and right margins
]{geometry}


%% If you want to chage the formating of the section headers
%\renewcommand{\thesection}{...}

\swapcommands{\Lambda}{\varLambda}
\swapcommands{\Omega}{\varOmega}

%%%%%%%%%%%%%%%%%%%%%%%%%%%%%%%%%%%%%%%%%%%%%%%%%%%%%%%%%%%%%%%%%%%%%%
\begin{document}%% v v v v v v v v v v v v v v v v v v v v v v v v v v
%%%%%%%%%%%%%%%%%%%%%%%%%%%%%%%%%%%%%%%%%%%%%%%%%%%%%%%%%%%%%%%%%%%%%%




\section{Numerical implementation of the Landau-Pollack-Slepian
  Theory}
\newcommand{\varD}{\ensuremath{\mathcal{D}}}

In the previous section, we saw that the teoretical limit, to how much
of the energy in a function bandlimited to the frequency region
$\Omega$ can be conentrated in the real space region \varD, is set by
an integral equation of the form 
\begin{equation} \label{eq:int-eig}
\lambda \Psi(\kappa) 
= \oldint_{\Omega} \rd{k}\, K_\varD(\kappa, k) \Psi(k)
\qcomma \kappa\in\Omega.
\end{equation}
This section will provide a method to solve this type of integral
eigenvalue equation numerically.

\subsection{Discretization of the integral equation}
The integral is discretized by
\begin{equation} \label{eq:inttosum}
\begin{aligned}
\oldint_\Omega\rd{k} &\to \sum_{n=1}^N\Delta{k}
=\frac{\abs{\Omega}}{N} \sum_{n=1}^N,\\
\Psi(k) \to \Psi_n = \Psi(k_n) &\qcomma
K_\varD(\kappa, k) \to K_{m, n} = K_\varD(k_m, k_n).
\end{aligned}
\end{equation}
where $\abs{\Omega}$ is the size of the region $\Omega$.
%Next, $\Psi$ and $K_\varD$ are discretized via
Note that $\kappa \to k_m$ (and not $\kappa_m$) since both 
$k, \kappa\in\Omega$, and thus they are both discretized in the same
way. 
With these discretizations, \eqref{eq:int-eig} discretizes to
\begin{equation}
\lambda\Psi_m = \frac{1}{N} \sum_{n=1}^N K_{m, n} \Psi_n
\end{equation}
which is just a matrix eigenvalue equation
\begin{equation} \label{eq:mtx-eig}
\Lambda \Psi = \mathsf{K}\Psi,
\end{equation}
where $\Lambda = N\lambda$ and $\mathsf{K}$ is the $N\times N$ matrix
whose element $(m, n)$ is $K_{m, n}$. From here, there are many high
performance linear algebra libraries to find the eigenvalues and
eigenvector to \eqref{eq:mtx-eig} numerically. 

\subsubsection{Physical interpretation}
\todo[inline]{Delta functions...}

\subsubsection{Improving this method using Simpson's rule}
\todo[inline]{\url{https://math.stackexchange.com/questions/898087/discretization-of-an-integral}\\
Will this even work?}
The main bottle neck in the numerical solution of \eqref{eq:int-eig}
is the eigenvalue and eigenvector calculation. It would therefore be
desirable to keep $N$ small. On the other hand, to accurately
approximate the integral using \eqref{eq:inttosum}, $N$ has to be
large. 
%This problem can be handled by changing the discretization method of
%the integral. 
The simple Riemann sum approximation, \eqref{eq:inttosum}, is one of
the most rudimentary integral discretizations. A good way to improve
numerical accuracy without sacrificing computation time in the
eigenvalue calculations, would be to use a better discretization
method. 

One such method is Simpson's rule (some times also refered to as
Simpson's $\nicefrac13$ rule)
\begin{equation}
\int_{k_{n-1}}^{k_{n+1}}f(k)\id{k} \approx 
\frac{\Delta{k}}{3} \Big[f(k_{n-1}) + 4f(k_n) + f(k_{n+1})\Big].
\end{equation}









%%%%%%%%%%%%%%%%%%%%%%%%%% The bibliography %%%%%%%%%%%%%%%%%%%%%%%%%%
%\newpage
%% This bibliography ueses BibTeX
\bibliographystyle{ieeetr}
\bibliography{references}%requires a file named 'references.bib'
%% Citations are as usual: \cite{example_article}

%%%%%%%%%%%%%%%%%%%%%%%%%%%%%%%%%%%%%%%%%%%%%%%%%%%%%%%%%%%%%%%%%%%%%%
\end{document}%% ^ ^ ^ ^ ^ ^ ^ ^ ^ ^ ^ ^ ^ ^ ^ ^ ^ ^ ^ ^ ^ ^ ^ ^ ^ ^ ^
%%%%%%%%%%%%%%%%%%%%%%%%%%%%%%%%%%%%%%%%%%%%%%%%%%%%%%%%%%%%%%%%%%%%%%




%%%%  Some (useful) templates


%% På svenska ska citattecknet vara samma i både början och slut.
%% Använd två apostrofer: ''.


%% Including PDF-documents
\includepdf[pages={1-}]{filnamn.pdf} % NO blank spaces in the file name

%% Figures (pdf, png, jpg, ...)
\begin{figure}\centering
\centerline{ % centers figures larges than 1\textwidth
\includegraphics[width=.8\textwidth]{file_name.pdf}
}
\caption{}
\label{fig:}
\end{figure}

%% Figures from xfig's "Combined PDF/LaTeX"
\begin{figure}\centering
\resizebox{.8\textwidth}{!}{\input{file_name.pdf_t}}
\caption{}
\label{fig:}
\end{figure}


%% If you want to add something to the ToC
%% (Without having an actual header in the text.)
\stepcounter{section} %For example a 'section'
\addcontentsline{toc}{section}{\Alph{section}\hspace{8 pt}Labblogg} 

